% This document is part of the Oddity project.
% Copyright 2015 the Authors.

% ## to-do
% - write

\documentclass[12pt]{article}
\newcommand{\given}{\,|\,}

\title{generalising and extending our use of DirMult for source finding}
\author{frean, hogg, brewer}
\date{}
\begin{document}\sloppy\sloppypar
\maketitle

Stuff here about stuff.

Goal: automatically find all the sources in a series of astronomical images, in
a way that's robust and fast per image, and which gets better the more images you process.

\section{the way we approach source-finding}

Generative model of the sky:
One can derive a score for
how well these locations match regions of ``non-background'' pixel
intensities (and potentially other features), and treat source-finding
as an exercise in optimizing those parameters.

For $k$ categories (bins) we will have a $k$-elt vector $\bn$ of
``counts'', for each class.  


\end{document}
